\section{Objective and Scope of the Thesis}
\label{sec:objective}

The aim of this thesis is to apply convolutional neural networks to computed tomography scans of porous 
rock samples to increase the quality of reduced exposure images by reducing the associated noise. 
In other words, training the proposed network by mapping low-quality images to higher-quality images, 
the network is expected to act as a denosing filter once applied to a given low-dose scan. 
There are several advantages of reducing the scanning time. First, it causes an increase in 
temporal resolution meaning even though the exposure time is lessened, higher-quality images 
can be obtained. Thus, capturing the dynamics of certain type of experiments can be possible 
because scan time is reduced. Secondly, image quality has a strong effect on the accuracy of 
the estimated rock properties in digital rock physics applications. 
With the help of the trained network, noise reduction is accomplished in the low-quality images. 
This will also help in the image processing steps like resolution enhancement. 



 
